\begin{intro}
Gracias al avance tecnol\'ogico en radioastronom\'ia y en almacenamiento de datos, el estudio de objetos estelares est\'a creciendo en Chile y en el mundo. D\'ia a d\'ia se producen grandes vol\'umenes de datos: s\'olo en ALMA (Atacama Large Millimeter Array) se producen 250 Terabytes anuales, y eso es poco en escala, considerando que en ASKAP (Australian Square Kilometre Array Pathfinder) se producen 2.5 Gbytes por segundo (75 Petabytes anuales). Toda esta informaci\'on producida en masa es guardada en distintas bases de datos alrededor del mundo, por ejemplo, grandes cantidades de objetos celestes se registran en SIMBAD (Set of Identifications, Measurements, and Bibliography for Astronomical Data), una base de datos de objetos externos al Sistema Solar. 

Esta enorme cantidad de datos necesitan ser interpretados por astr\'onomos y otros cient\'ificos para poder avanzar en nuestro entendimiento del universo. Es por esto que surge la necesidad de contar con t\'ecnicas y herramientas computacionales que faciliten su estudio y an\'alisis. El poder visualizar datos de tipo astron\'omico de forma intuitiva y coherente es un aspecto clave para poder reconocer patrones y realizar ciencia con los mismos. Otro aspecto importante al momento de estudiar datos astron\'omicos es la posibilidad de filtrarlos a trav\'es de distintos criterios, con el fin de poder trabajar sobre un subconjunto de ellos con caracter\'isticas espec\'ificas.

Actualmente, para guardar la informaci\'on obtenida en las observaciones astron\'omicas como imagen se utiliza el formato \textbf{FITS} (Felxible Image Transport System). Este formato permite guardar arreglos multidimensionales de datos (espectros en una dimensi\'on, im\'agenes en dos dimensiones o cubos de datos en 3 o m\'as dimensiones) y un encabezado con la metadata pertinente a la im\'agen, por lo que es ampliamente utilizado en la comunidad cient\'ifica. Las im\'agenes guardadas en este formato son en general de gran tama'no, por lo que un monitor de computador de escritorio (por ejemplo, 1080p) no es suficiente para desplegarla completa en una raz\'on 1:1 entre los pixeles del monitor y los pixeles de la im\'agen. Esto puede traducirse en un an\'alisis menos preciso.


\section*{Motivaci\'on}
La fundaci\'on INRIA Chile (en particular el equipo Massive Data, liderado por Emmanuel Pietriga), en conjunto con el Departamento de Astronom\'ia de la Universidad Cat\'olica de Chile, desarroll\'o un software de visualizaci\'on de im\'agenes astron\'omicas en formato FITS. Dado el tama'no de tales im\'agenes, la aplicaci\'on est\'a pensada para correr en un wall-display de alta resoluci\'on. Este software, que fue desarrollado como primer prototipo, es capaz de desplegar la imagen y de aplicarle a esta distintos filtros de color. Sin embargo, ese software no permite realizar una selecci\'on de un subconjunto de datos con características específicas ni permite la b\'usqueda de alg\'un objeto particular. Resulta interesante, por ejemplo, dado un punto y un radio, encontrar todos los objetos celestes contenidos en la vecindad. Adem\'as ser\'ia \'util desplegar informaci\'on sobre un objeto en particular una vez encontrado, como su velocidad radial, brillo y visualizar estos resultados en paralelo con la imágen astron\'omica.
Bajo este contexto, se implementa sobre la aplicaci\'on prototipo ya existente una interfaz gr\'afica que permita al usuario hacer consultas a la base de datos astron\'omica SIMBAD, que almacena informaci\'on acerca de objetos astron\'omicos m\'as all\'a del Sistema Solar. Una lista de los par\'ametros de las consultas y de los posibles campos en los resultados se encuentra documentado en la página de ayuda de SIMBAD.

 La interfaz adem\'as presenta al usuario los resultados encontrados de forma intuitiva y \'util. Para desarrollar lo anterior, se utiliza ZVTM , una herramienta implementada en Java de desarrollo de interfaces de usuario que permite trabajar con grandes cantidades de datos eficientemente y que soporta el uso de wall-displays (un arreglo de pantallas manejado por un clusters de computadores), aunque tambi\'en es ejecutable sobre un computador con una s\'ola pantalla. ZVTM est\'a hecho para trabajar con visualizaciones de datos en 2D y se basa en la met\'afora de tener varios espacios virtuales a modo de universos separados, cada uno puede ser observado con una o m\'as c\'amaras m\'oviles; de esta forma se logra continuidad en la percepci\'on de animaci\'on de objetos y movimientos de c\'amara. En esta herramienta cada objeto es representado como un \textit{glyph} (objeto renderizable) que puede personalizarse. ZVTM puede desplegar im\'agenes, vectores gr\'aficos, texto, aplicaciones y hasta documentos PDF. Para correr ZVTM en un wall-display se utiliza la librer\'ia jBricks, que facilita el desarrollo, pues se encarga de coordinar la renderizaci\'on entre las pantallas.

La aplicaci\'on est\'a hecha para correr en un wall-display debido a su potencial en atronom\'ia, por lo que la interfaz se prueba en el ANDES wall-display ubicado en las oficinas de la fundaci\'on INRIA Chile y se adapta a esta forma de interacci\'on con los datos para hacer uso provechoso su potencial. El ANDES wall-display está compuesta por 24 paneles LED táctiles (la resoluci\'on es de 11520 x 4320 pixeles) manejados por un cluster de 13 computadores (12 de ellos son coordinados por el n\'umero trece).

\end{intro}